In this thesis work, we presented \name -- an open source framework for empirical research of RSP Engines. \name aims at enabling the Systematic Comparative Research Approach in the Stream Reasoning field, through an RSP Engine \textsc{Test Stand}, four Baselines of RSP Engines, and an \textsc{Analyser}. 

The motivations that led this work are included in Chapter~\ref{chap:problem-settings}, while in Chapter~\ref{chap:heaven} we described \name design and in Chapter~\ref{chap:implementation-experience} we detailed the implementation of the \textsc{Test Stand}, the four Baselines and the \textsc{Analyser}. Finally, we presented an empirical proof of \name potential in Chapter~\ref{chap:evaluation}, providing examples of Experiment Design, two Test sets (SOAK and Step Response), and an evaluation of \name Baselines, which exploits them.

We learnt that, even when RSP Engines are extremely simple (e.g., the baselines), it is hard to demonstrate hypothesis formulated only from a theoretical knowledge and, thus, empirical evaluation is required. Experiment results emphasised the importance of conducting comparative research based on controlled experimental conditions. Thus, we confirmed that the SR community needs an open source\footnote{\url{https://github.com/streamreasoning/HeavenTeststand}} framework like \namens.

The focus on the experimental infrastructure is the main difference between \name and previous works. While SRbench and LSbench focus on RDF Streams and a suite of continuous SPARQL queries, \name allows to compare RSP Engines based on any RDF Stream, ontology, continuous query and entailment regime. It even enables to run experiments connecting to live data streams as those used in~\cite{DBLP:conf/semweb/BalduiniVDTPC13}.	

In this chapter, we recap this thesis works, presenting in Section~\ref{sec:research-question-conclusion} our Research Question and a brief description of \name design and implementation, which answers such research question. Last but not least, in Section~\ref{sec:research-fw-conclusion} we point out \name limitations and the future works of this thesis.

\section{Comparative Research of RSP Engines}\label{sec:research-question-conclusion}

Stream Processing research field is growing and the number of techniques to semantically handle data stream is increasing. RDF Stream Processing Engines, a.k.a. RSP Engines, are systems able to answer continuous extensions of SPARQL queries over RDF Streams. Due to their complexity, it is hard to systematic compare them under repeatable conditions,. 

It is worth to note that, despite the Engineering epistemology of the Computer Science works, it is still present the lack of a Systematic Comparative Research Approach (SCRA)~\cite{Tichy:1995:EEC:209090.209093}. SCRA is typical of those research areas which have to face very complex systems, and have difficulty to simplify the models. Architectural analysis are useful, but they are not sufficient to evaluate RSP Engines, because their behaviour must be studied during the execution. For this reason, the Stream Reasoning community has tried to define and develop solutions to evaluates RSP Engines~\cite{DBLP:conf/esws/ScharrenbachUMVB13}. Recent works like~\cite{Zhang2012, LePhuoc2012c, DBLP:conf/semweb/DellAglioCBCV13} supported this approach with queries, dataset and methods. However the SR community still lacks an experimental infrastructure which enables the comparison of RSP Engines independently from RDF Stream, ontology, continuous query and entailment regime.  From aerospace engineering we borrow the idea of engine test stand: a facility to develop engine through systematic testing under precise experimental conditions. Thus, we can formulate our research question as follow:\\

\textit{”Can an engine test stand, together with queries, datasets and methods, support Systematic Comparative Research Approach for Stream Reasoning?”}\\

%esperiment, reproducibility, repetability and comparability to enable the SRCA by a test stand
%simple terms of comparison to support the comparative research: SERE properties
%test stand design and implmentation fulfill the requirements posed
%Analyser consist in the definition of an analysis stack which extend% the traditonal top-down investiagtion based on hypothesis formulated on the model, through empirical findings ad different level of analysis.
%We prove the value of the empirical resarch heaven test stand enables by evaluating the baselines
%we see that even for simple system like the baselines, which model is known, upredictable results may rise
%now it is possible to improve existing RSP engine model through finding provide by the different analysi levels of the stack

\noindent In the following we provide an evidence of how \name positively answers the research question.

In Chapter~\ref{chap:problem-settings}, we describe how an engine test stand must be in the Stream Reasoning research field. We exploit the traditional experiment definition to formulate the requirements that a Test Stand for RSP Engine must fulfil, in order to answer our research question and to grant the rigorous and systematic test of RSP Engines. SCRA, due to its case-oriented nature, demands simple terms of comparison, namely baselines, to exploit for initial evaluation examples. In Chapter~\ref{chap:problem-settings}, we detail which properties a baseline must have and we formulate them as requirements for their implementation.

%Three main experiment properties, \textit{reproducibility}, \textit{repeatability} and \textit{comparability}, allow us to formulate the requirements that a Test Stand for RSP Engine must fulfil, in order to answer our research question.

In Chapter~\ref{chap:heaven}, we describe \name design. We explain how the \textsc{Test Stand} and the Baselines should be to fulfil the requirements we posed. We introduce also the idea of the \textsc{Analyser} as an investigation stack that extends the research of RSP Engines from the traditional hypothesis based approach to the empirical and comparative one. In Chapter~\ref{chap:implementation-experience}, we describe how \name \textsc{Test Stand} and the Baselines are implemented. We also show how we realised the investigation stack, providing a statistical evaluation of experiment results at higher levels, while the lower ones offer an overview of the RSP Engine dynamics over all the experiment execution.

%Our research question asks to demonstrate if a \textsc{Test Stand} for RSP Engine can support the SCRA for Stream Reasoning. 
In Chapter~\ref{chap:evaluation}, we show how the traditional top-down analysis are not enough for evaluating complex systems like RSP Engines, even in case of naive implementations. Our evaluation exploits an experimental set composed by SOAK Tests and Step Response Stress Tests, executed on the Baselines implementations that we included in \name framework. The results of the analysis show how the traditional research, which formulate hypothesis only on the RSP Engine model knowledge, is still meaningful, but it can be improved through an infrastructure like \name \textsc{Test Stand}. The evaluation conducted in Chapter~\ref{chap:evaluation} has shown that it is hard to demonstrate even naive hypothesis. RSP Engine dynamics can be only partially investigated from the statistical viewpoint. We need further knowledge about the RSP Engine dynamics, which means observing their behaviour at once and over the entire execution of an experiment. %In this way it is possible to apply SCRA at any details level it requires.\\

%Works like~\cite{DBLP:conf/esws/ScharrenbachUMVB13,DBLP:conf/semweb/DellAglioCBCV13} defined how to benchmark RSP Engine systems~\cite{DBLP:conf/esws/ScharrenbachUMVB13} w.r.t many existing and limited proposals~\cite{Zhang2012, LePhuoc2012c}. 

\name allows to drill down the analysis over an investigation stack which covers all the aspects of the dynamic system performance analysis. Through \name is now possible to improve existing theoretical models thanks to the empirical findings that were not available before. Thus, we can positively answer our research question, stating that \name sustains SCRA and extends the traditional top-down analysis. 

\section{Limitations And Future Works}\label{sec:research-fw-conclusion}

During \name development, we faced many issues related to the heterogeneous nature of RSP application domains. These concerns limit our work in different ways. They influence \name development in term of both design and implementation. Moreover, our research of RSP Engines  is actually restricted to \name Baselines within an extremely controlled experimental setting.

The limitations on \name design and its implementation must be faced, improving its models and further developing the current implementation of the \textsc{Streamer}, the \textsc{ResultCollector} and the \textsc{Test Stand External Structure}. On the other hand, the restrictions on the research of RSP Engines require to exploit \name \textsc{Test Stand} in order to pursue the analysis. Finally, we consider a further possible contribution continuing the research on the Baselines, which has its own scientific value, as Chapter~\ref{chap:evaluation} partially evidenced.

Due to these limitations, the future works and possible extensions of \name belong to the following categories:
\begin{itemize}
\item \textit{Research of RSP Engine} - it involves the empirical evaluation of RSP Engine and the comparison of benchmarking results, which are our main research interests. Thus, we plan to support our research through \namens.
\item \textit{Software Engineering and Development} - it involves future works focus on the different aspects of \name software, which is extendible by design.
\item \textit{Research on Baselines} - it aims to provide a complete evaluation of the Baselines as simple terms of comparison for mature RSP Engines.
\end{itemize}

\noindent We aim extending the \textit{Research Work}, creating a ready-to-use benchmarking suite built upon \namens, which allows to test any RSP Engine with a set of well defined experiments. From preliminary studies we know that which is the essential test set to cover the most important uses cases.\\

An essential set of experiments must include the following tests:
\begin{itemize}
\item \textbf{T1} SOAK.
\item \textbf{T2} Stress Step.
\item \textbf{T3} Stress Sine Wave.
\item \textbf{T4} Poisson Distribution.
\end{itemize}

The experiments definition still follows the tuple $<\mathcal{E},\mathcal{D},\mathcal{T},\mathcal{Q}>$. The ready-to-use benchmarking suite will give the users the possibility to execute those test on their own RSP Engines. Moreover, it should include \name Baselines as simple terms of comparison for benchmarking results.

SOAK Test [T1] and Stress Step Test [T2] are already part of this thesis work in a restricted form, while the other ones are not implemented yet. 

In the current stage of development, it is possible to configure only the ontology and the entailment regime of the Baselines. Thus we develop [T1] and [T2] registering to our $\mathcal{E}$ as queries $\mathcal{Q}$ variations of the identity query, which differ for window size $\omega$. We intend to continue the development of the Baselines, adding the possibility to register one or more continuous queries into them and exploiting more complex entailment regime than $\rho$DF. Moreover, we have to define which queries $\mathcal{Q}$ to include in all the experiments, considering many works in the field~\cite{DBLP:conf/esws/ScharrenbachUMVB13, Zhang2012, LePhuoc2012c, DBLP:conf/semweb/DellAglioCBCV13}.

The current experiment sets [T1] and [T2] exploit LUBM ontology as $\mathcal{T}$ and the RDF Stream $\mathcal{D}$ is generated through a module, the \textsc{RDF2RDFStream}, which adapts LUBM data to a streaming scenario (See Chapter~\ref{chap:implementation-experience}). In order to generate data for all the remaining test sets [T3] and [T4] we have to extend the \textsc{RDF2RDFStream} to generate: a random flow with a Poisson distribution, and a sine wave flow (to mimic the periodic changes in the flow rates observed on social media streams~\cite{DBLP:conf/semweb/BalduiniVDTPC13}).

Independently from the experiment set, we aim to extend the \textsc{Test Stand} measurement as suggested in~\cite{DBLP:conf/esws/ScharrenbachUMVB13}:
\begin{itemize}
\item Response time over all queries (Average/$1^{th}$ Percentile/Maximum).
\item Maximum input throughput in terms of number of data element in the input stream consumed by the system per time unit.
\item Minimum latency to accuracy and minimum latency to completeness for all queries.
\end{itemize}

As we stated in Chapter~\ref{chap:evaluation}, the content of the active window influences the RSP Engine performances for two factors: the window size before the reasoning and after. The second metrics is relevant for the RSP Engine evaluation. When the system has to handle a big number of outgoing triples we can observe a degradation in terms of memory and latency. Thus, an evaluation of the number of inferred triples w.r.t the window content at time $t$ allows to weight the engine performance results in relation to the input RDF Stream, helping to eliminate outliers and properly evaluate RSP Engines. Moreover, this observation opens new scenarios in the Stress Testing design, where the stress factor depends on the reasoning potential of the current window w.r.t a certain entailment regime.\\

\noindent Future works on \textit{Software Engineering and Development} regard the \textsc{Analyser}.  We aim to completely automate the analysis procedure, involving at least the current measurement set and tools. 

As a long term goal, we intend to standardise the entire tool-set which supports analysis methods presented in Chapter~\ref{chap:heaven}. We are imagining the \textsc{Analyser} as a Web-based environment where all existing RSP Engines are already available; the experiments design and execution are accessible as a service through the selection of RDF Streams, ontologies, and queries; the results analysis of any external RSP Engine is allowed providing the data and the involved variables. Finally, a visual facility to compare different experiments and the publication of experiment result as linked data would complete this environment. \\


\noindent Last but not least, we would like to continue the \textit{Research on Baselines}, because we identified an intrinsic scientific value in their evaluation, which can be another contribution itself. In order to study the problem of responsiveness we have to add four alternative implementations of the Baselines, which do not exploit the external time control. The Baselines should be evaluated by [T1], [T2], [T3] and [T4] but also with real data and a $\mathcal{T}$ different from LUBM, for example exploiting LS Bench queries and data to design an experiment set. 
