Lo Stream Reasoning \`e  il settore di ricerca che ha dimostrato la possiblit\`a di applicare procedure di reasoning su flussi informativi in rapido cambiamento. Un RDF Stream Processing (RSP) Engine \`e  un sistema in grado di processare a livello semantico questi flussi, quando sono codificati secondo lo standard RDF. Il numero di RSP Engine implementati \`e  in crescita e di conseguenza la comunit\`a scientifica sta formalizzando i metodi e gli strumenti che hanno consentito lo sviluppo di queste soluzioni.

Diversi settori di ricerca nell'ambito della Computer Science, hanno mostrato interesse per una maggiore comprensione della natura del proprio lavoro. Sono stati fatti diversi studi che hanno analizzato i frutti della ricerca in questi settori~\cite{Tichy:1995:EEC:209090.209093, Wainer:2009:EEC:1518331.1518552}. Questi hanno dimostrato anzitutto la natura ingegneristica di molte pubblicazioni nell'ambito della Computer Science, ma anche una discreta mancanza di valutazioni empiriche delle soluzioni implementate. Questa \`e  una differenza evidente con le altre aree di ricerca legate al mondo dell'ingegneria, che si focalizzano su questo tipo di analisi.

Solitamente, nei settori informatici in cui la valutazione empirica \`e tralasciata, i sistemi proposti hanno una natura complessa e sfaccettata che \`e difficile da valutare. Tuttavia \`e  possibile, con gli strumenti adatti, studiare anche casi complessi. Questo accade per le scienze sociali o l'economia, i cui soggetti d'indagine non sono di certo facilmente modellabili. In questi settori viene comunemente usato un approccio comparativo sistematico, che semplifica il problema di affrontare soggetti complessi, senza tralasciare gli aspetti che li rendono rilevanti. Questo approccio diventa applicabile solo in un contesto sperimentale appropriato, che garantisce propriet\`a come riproducibilit\`a, ripetibilit\`a e comparabilit\`a.

La comunit\`a dello Stream Reasoning ha colto la necessit\`a di fornire strumenti per valutare correttamente gli RSP Engine,  comprenderne il comportamento e quantificarne il valore comparando le prestazioni in casi d'uso reali. Qualche passo in questa direzione \`e  stato gi\`a fatto. Lavori recenti~\cite{Zhang2012, LePhuoc2012c, DBLP:conf/semweb/DellAglioCBCV13} hanno fornito framework di benchmarking per RSP Engine, mentre altri hanno posto le basi di queste valutazioni~\cite{DBLP:conf/esws/ScharrenbachUMVB13}, mostrando quali erano le mancanze di tali framework.

Le soluzioni proposte si sono dimostrate limitate, e la valutazione empirica di RSP Engine \`e  solo all'inizio. Quello che ancora manca \`e  una infrastruttura che permetta la comparazione sistematica di RSP Engine, all'interno di un contesto sperimentale che goda delle proprit\`a sopracitate. Per affrontare il problema, in questa tesi, abbiamo preso dall'ingegneria aerospaziale l'idea di un banco di prova, uno strumento di valutatione e sviluppo per motori.

Un banco di prova permette di progettare esperimenti ed eseguirli su qualsiasi motore, raccogliendo i dati per una successiva valutazione delle prestazioni. 
La nostra domanda di ricerca quindi \`e : "Un banco di lavoro per RSP Engine \`e  la soluzione che permetta la ricerca comparativa e sistematica nell'ambito dello Stream Reasoning?"

In questa tesi proponiamo \namens, un framework open source per la ricerca comparativa e sistematica nell'ambito dello Stream Reasoning. Il framework si compone di un Banco di Lavoro, l'equivalente di quanto abbiamo visto nell'ingegneria aerospaziale ma per RSP Engine. Include quattro implementazioni naive di RSP Engine, dette Baselines. Questi sistemi semplificati permettono di iniziare la ricerca comparativa. Infine \name contiene l'Analyser, un insieme di metodi di indagine e strumenti di supporto, organizzati gerarchicamente ed atti ad analizzare e comparare i dati raccolti attraverso l'esecuzione di esperimenti su RSP Engine tramite il banco di lavoro.
