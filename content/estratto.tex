Lo Stream Reasoning \`e  il settore di ricerca che ha dimostrato la possiblit\`a di applicare procedure di reasoning su flussi informativi in rapido cambiamento. Un RSP Engine \`e  un sistema in grado di processare a livello semantico questi flussi, quando sono codificati secondo lo standard RDF. Il numero di RSP Engine implementati \`e  in crescita e di consequenza la comunit\`a scientifica sta formalizzando i metodi e gli strumenti che hanno consentito lo sviluppo di queste soluzioni. A questo punto \`e  necessario fornire strumenti di valuazione di queste soluzioni, per comprenderne il comportamento e valutarne le performance in casi d'uso reali. 
Qualche passo in questa direzione \`e  stato gi\`a fatto. Lavori recenti \cite{Zhang2012, LePhuoc2012c, DBLP:conf/semweb/DellAglioCBCV13} hanno fornito framework di benchmarking per RSP Engine, mentre altri hanno posto le basi di queste valutazioni \cite{DBLP:conf/esws/ScharrenbachUMVB13}, mostrado quali erano le mancanze di tali framework.

Diversi settori di ricerca nell'ambito della Computer Science, hanno mostrato interesse per una maggiore comprensione della natura del proprio lavoro. Sono stati fatti diversi studi che hanno analizzato i frutti della ricerca in questi settori \cite{Tichy:1995:EEC:209090.209093, Wainer:2009:EEC:1518331.1518552}. Questi hanno dimostrato anzitutto la natura ingegneristica di molti lavori nell'ambito della Computer Science, ma anche una discreta mancanza di valutazioni empriche delle soluzioni implementate. Aree di ricerca diverse, sempre legate al mondo dell'ingegneria, si focalizzano su questo tipo di valutazioni. Nel contesto informatico questo non avviene, a causa della natura complessa e sfaccettata dei sistemi proposti. Tuttativa \`e  possible con gli strumenti adatti, valutare anche casi complessi. Questo accade per le scienze sociali o l'economia, i cui soggetti d'indagine non sono di certo facilmente modellabili.

Anche la ricerca nell'ambito dello Stream Reasoning ha mostrato le stesse mancanze. Le soluzioni proposte si sono dimostate limitate, e la valutazione di RSP Engine \`e  solo all'inizio. Quello che ancora manca \`e  una infrastruttura che permetta la comparazione sistematica di diverse soluzioni implementate. Dall'ingegneria aerospaziale abbiamo preso l'idea di un banco di prova. Uno strumento di valutazione, che permette di progettare esperimenti ed eseguirli su qualsiasi motore, valutandone le prestazioni. %Dal punto di vista scientifico un esperimento gode di tre proprit\`a fondamnetali: riproducibilit\`a, ripetibilit\`a e comparabilit\`a, tre pilastri su cui \`e  possible costrurie un approccio comparativo sistematico per la ricerca. 
La nostra domanda di ricerca quindi \`e : "Un test stand per RSP Engine \`e  la soluzione che permetta la ricerca comparativa e sistematica nell'ambito dello Stream Reasoning?"

In questa tesi proponiamo \namens, un framework open source per la ricerca comparativa e sistematica nell'ambito dello Stream Reasoning. Il framework si compone di un Test Stand, l'equivalente di quanto abbiamo visto nell'ingegneria aerospaziale ma per RSP Engine. Include quattro implementazioni naive di RSP Engine, dette Baselines, che sistemi semplificati da cui iniziare la ricerca comparativa. Infine \name contiene l'Analyser, un insieme di metodi di indagine e strumenti di supporto atti ad analizzare e comparare i dati di valutazione degli RSP Engine, sfruttando un sistema di indagine organizzato gerarchicamente.
